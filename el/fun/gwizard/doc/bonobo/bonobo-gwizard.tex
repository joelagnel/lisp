%
% bonobo-easy.tex
%

% includes
\documentclass[final,10pt]{article}
\usepackage{verbatim}
\usepackage[breaklinks,colorlinks]{hyperref}
\usepackage{simplemargins}
\usepackage{epsfig}
\usepackage{html}

% macros
\newcommand{\link}[1]{\htmladdnormallink{#1}{#1}}
\begin{latexonly}
\renewcommand{\link}[1]{\href{#1}{#1}}
\end{latexonly}
\newcommand{\gwizard}{{\tt gwizard}}

\parskip 2mm                  
\parsep 2mm 
\parindent 0mm

\title{Writing Bonobo Components: the easy way}
\author{Dirk-Jan C. Binnema, 
  {\link{http://www.djcbsoftware.nl}}\\
  dirk-jan@djcbsoftware.nl}
\begin{document}
\maketitle
\begin{abstract}
  This article shows how creating a Bonobo components with
  both {\em multiple} and {\em custom} interfaces is not hard at all,
  using the \gwizard{} emacs-lisp macros.
\end{abstract}

\section{Introduction}
There are some introductions to Bonobo that show some basic
things. Up until now, there's not been a more advanced text 
(that I know of). I hope
this article can help you to take the next step. If you don't know
the basics of Bonobo, please read \cite{bonobo-nluug} (technical overview),
\cite{ipctrl} (how-to) or \cite{meeks1} first. Also, basic
(working) knowledge of CORBA is assumed, and finally, you should know
how to operate Emacs, because the \verb|gwizard| tool I'll be using is
Emacs based.

{\bf Big warning}: A wizard may help you save some typing, but should never be a
substitute for knowing what you are doing!

In this article I'll discuss the writing of a Bonobo component which has both
\begin{itemize}
        \item multiple interfaces, and 
        \item custom interfaces. 
\end{itemize}
Actually, the same interfaces are both multiple and custom... 

Creating Bonobo components is a lot easier with the right tools. Well,
to save you from a lot of boring boilerplate code and RSI, I have
written \verb|gwizard|, a set of Emacs elisp-macros~\cite{gwizard}.

\section{Components have feelings too}
Let's introduce our example component: \verb|MoodyComponent|. 
\verb|MoodyComponent| is a, well, moody component: 
sometimes, it's in a good mood, and sometimes, it's in a bad mood.

We can model the moods of \verb|MoodyComponent| as separate
interfaces: one interface for the good mood, and one interface for the
bad mood. Figure~\ref{fig:moody-component} gives an artist's
impression of this.
\begin{figure}[ht]
  \centerline{%
    \epsfxsize=80mm
    \epsffile{moody-component.eps}}
  \caption{Component exposing multiple interfaces}
  {\label{fig:moody-component}}
\end{figure}

\subsection{Writing the IDL}
CORBA IDL is the language we use for the definition of the
interfaces. We won't discuss CORBA IDL here any further; there are
many places you can learn about that, for example \cite{HV99}, 
and of course there's always 
the official specification from the OMG \cite{omg}.

Note that I put the interfaces in
the \verb|Bonobo::Sample::| namespace. Also note that we're writing
Bonobo interfaces, so by definition they derive (directly or
indirectly, is this case directly) from \verb|Bonobo::Unknown|.
Figure~\ref{fig:moods-uml} gives a UML-like class diagram.
\begin{figure}[ht]
  \centerline{%
    \epsfxsize=80mm
    \epsffile{moods-uml.eps}}
  \caption{MoodyComponent class diagram}
  {\label{fig:moods-uml}}
\end{figure}
We can now express these interfaces in CORBA IDL. To prevent the
polution of the IDL namespace, we place our Moody interfaces
in the \verb|Bonobo::Sample| namespace. Likewise, for the 
\verb|.idl|-file, we choose the name \verb|Bonobo_Sample_Moody.idl|
(it would be stylistically ugly to call it
\verb|Bonobo_Sample_Moody_Component.idl|, because we'd like to maintain
the separation of interface and implementation).

The actual IDL is very simple, and looks like this:
\begin{verbatim}
/*
 *  Bonobo_Sample_Moody.idl
 */

#include <Bonobo.idl>

module Bonobo {
        module Sample {
                interface GoodMood : Bonobo::Unknown {
                        string say_hello ();        
                };
        
                interface BadMood : Bonobo::Unknown {
                        string say_hi ();
                };
        };
};

\end{verbatim}
If you wonder why \verb|GoodMood| has \verb|say_hello| and
\verb|BadMood| has \verb|say_hi|, that's because I'd like to show they
are really different; if they had the same name, some people may
make the wrong assumptions.

In the example, we derive our interfaces directly from
\verb|Bonobo::Unknown|, because that's all we need. However, if we
wanted to build a control we could derive for \verb|Bonobo::Control|,
if we wanted to support structured storage we could derive from
\verb|Bonobo::Storage| etc.

\section{Implementation using Bonobo}
\subsection{Background: interfaces and implementation}
CORBA IDL-interfaces are really all about {\em interface inheritance}:
they specify what methods an interface support, but leave the
implementation of those methods up to {\bf you}. This holds also true
for the Bonobo IDL, as they {\bf are} CORBA IDL interfaces like any
other.

However, together with the {\bf interfaces} hierarchy, Bonobo comes with an
{\bf implementation} hierarchy that provides default implementations for the
methods in the Bonobo interfaces. In the Bonobo source distribution, you'll
find the IDL-interfaces in the \verb|idl/| directory, while the
default implementations are in \verb|bonobo/|

The mapping of IDL-interfaces upon implementation is mostly quite
straightforward, for example the \verb|Bonobo::Print| interface has a
default implementation in \verb|BonoboPrint| and \verb|Bonobo::Stream|
finds its counterpart in \verb|BonoboStream|.  The notable exception here is
\verb|Bonobo::Unknown|, which is implemented by \verb|BonoboXObject|.

Also note that the implementations are done using the Gtk+ object
system: they're \verb|GtkObject|s in Gtk+/Glib/Gnome 1.x, but will be
\verb|GObject|'s in Gnome2. However, we're discussing only the former
here. 

\subsection{Generating the code}
After we have specified the interfaces, we can implement them. This is
quite easy, especially with \verb|gwizard|; after installation we can
do (in Emacs):
\begin{verbatim}
M-x gwizard-new-bonobo1-interface[RET]  
Interface: Bonobo::Sample::GoodMood[RET]
Parent: Bonobo::Unknown[RET]
Long names (y/n): n
License (1=GPL, 2=LGPL, 3=none): 1
\end{verbatim}
So what are we doing here? Well:
\begin{itemize}
  \item The 'Interface' is the interface we want to implement. This
  corresponds to the IDL interface, with full namespace qualification;
  \item The 'Parent' is the interface we're deriving from. We derive
    directly from \verb|Bonobo::Unknown|, so that's what we enter;
  \item The 'Long names' refers to the naming of the generated files and
  their contents for this interface; do you want
  \verb|bonobo-sample-good-mood.c| or just \verb|good-mood.c|? If you
  want the latter, say 'n' here;
  \verb Finally, \verb|gwizard| can automatically insert the GPL or
  LGPL license header if you want.
\end{itemize}

That leaves us with the boilerplate implementation code for the
\verb|GoodMood| interface in \verb|good-mood.h| and
\verb|good-mood.c|. I urge you to study the generated code; it's
more-or-less a normal \verb|GtkObject|, as described in
\cite{Pennington99}, so it shouldn't be too hard.

\subsection{Filling out the details}
\gwizard{} has done most of the boring work for use, but there are a couple
of things we must do to finish the \verb|GoodMood|-implementation:
\begin{enumerate}
  \item \verb|good-mood.h|: Add the include file that CORBA created from the
  \verb|Bonobo_Sample_MoodyComponent.idl|; 
  \item \verb|good-mood.[ch]|: Add the declarations and
  implementations for the \verb|say_hello| function in the
  \verb|GoodMood| interface;
  \item \verb|good-mood.c|: connect our object implementation to our
  entry point vector; i.e. tell CORBA where it can find our implementation.
\end{enumerate}

Let's do these things:

First, we add the include file with all the CORBA
definitions. Remember our \verb|.idl| is called
\verb|Bonobo_Sample_Moody.idl|; well, after IDL-compilation
(which we haven't done yet), this will produce a file called
\verb|Bonobo_Sample_Moody.h|, and that's the one we must
include here. So we add to \verb|good-mood.h|, right under \#include
that's already there:
\begin{verbatim}
#include "Bonobo_Sample_Moody.h"
\end{verbatim}

Ok, now for the \verb|say_hello| implementation, in \verb|good-mood.h|
we add:
\begin{verbatim}
CORBA_char*  good_mood_say_hello (PortableServer_Servant servant,
                                  CORBA_Environment * ev);
\end{verbatim}
(The signature can be deduced from the IDL$\to$C-mapping, or by
grepping the generated \verb|Bonobo_Sample_Moody.h| file for
\verb|say_hello| (you can
compile it by hand with \verb|orbit-idl|). 
The implementation in \verb|good-mood.c| is also very simple; add the
end of this file add:
\begin{verbatim}
static CORBA_char*
good_mood_say_hello (PortableServer_Servant servant,
                     CORBA_Environment *ev)
{
   return CORBA_string_dup ("Hi! How are you?");
}
\end{verbatim}

Now, the last thing we need to do is hooking uo our implementation up
with the so-called {\em entry point vector} (epv), in the
\verb|good_mood_class_init|:
\begin{verbatim}
    epv->say_hello = good_mood_say_hello;
\end{verbatim}
And that's it! The final step tells CORBA where to find our
implementation for the \verb|say_hello| function.

\subsection{What about the Bad Mood?}
Not surprisingly, the implementation for \verb|BadMood| is very similar:
\begin{verbatim}
M-x gwizard-new-bonobo1-interface[RET]  
Interface: Bonobo::Sample::GoodMood[RET]
Parent: Bonobo::Unknown[RET]
Long names (y/n): n
License (1=GPL, 2=LGPL, 3=none): 1
\end{verbatim}

I guess it's a nice exercise for the reader to implement the
\verb|BadMood|-interface without any help... Oh, remember you must
create a \verb|bad_mood_say_hi|-method:
\begin{verbatim}
CORBA_char*     
bad_mood_say_hi  (PortableServer_Servant _servant,
                  CORBA_Environment * ev)
{
        CORBA_string_dup ("Grmpff... stop bothering me!");
}
\end{verbatim}
Well, if you succeeded, we have now implemented the two
interfaces. Hurray!

\subsection{Writing the factory}
Now we have implemented the two interfaces. To turn them into a 
component, we need to have a 'factory', a piece of software that
produces components. Writing factories is very easy using the
\verb|gwizard-new-bonobo1-factory| macro in the \verb|gwizard|-
package:
\begin{verbatim}
M-x gwizard-new-bonobo1-factory[RET] 
Name: Bonobo::Sample::MoodyComponent[RET]
License (1=GPL, 2=LGPL, 3=none): 1
\end{verbatim}
This generates \verb|bonobo-sample-moody-component.c| and
\verb|Bonobo_Sample_MoodyComponent.oaf|. 

Finishing \verb|bonobo-sample-moody-component.c| is quite easy:
add \verb|#include|'s for \verb|good-mood.h| and \verb|bad-mood.h|,
and tie them together in the factory method:

\begin{verbatim}
static BonoboObject*
bonobo_sample_moody_component_factory (BonoboGenericFactory* factory,
                                       void* data)
{
        GoodMood *good_mood = good_mood_new ();
        BadMood  *bad_mood  = bad_mood_new ();
        
        bonobo_object_add_interface
                (BONOBO_OBJECT(good_mood), BONOBO_OBJECT(bad_mood));
       
        return BONOBO_OBJECT (good_mood);
}
\end{verbatim}
You must select one ``primary'' interface (\verb|good_mood|) in this
case, to which you can add other interfaces. It doesn't matter which
one you choose; just choose one.

\subsection{Finishing the .oaf}
Now, we've written all the code for the component, all that's left is
registering it with OAF. The generated \verb|.oaf| should mostly work,
but what you still need to do is add the interfaces you implement:
\begin{verbatim}
<oaf_attribute name="repo_ids" type="stringv">
         <item value="IDL:Bonobo/Sample/GoodMood:1.0"/>
         <item value="IDL:Bonobo/Sample/BadMood:1.0"/>
</oaf_attribute>
\end{verbatim}
There are some comments in the \verb|.oaf| that indicate where to put
this. 

You should also check the \verb|location| attribute of the Factory;
make sure it's either somewhere in your \verb|PATH| or fill in the
path to the component executable.

Place the \verb|.oaf| in your oaf-directory, which may be
\verb|/usr/share/oaf|, or anywhere in your \verb|OAF_PATH|. Check
the OAF-documentation \cite{liboaf} for details.

\section{Compiling}
Of course, we'd like to turn our work into some executable code. 
We write a little \verb|Makefile| (this is not generated, but
old-fashioned hand work):
\begin{verbatim}
#
# Makefile for bonobo-sample-moody-component
#

CORBA_GENERATED = \
        Bonobo_Sample_Moody-common.c \
        Bonobo_Sample_Moody-skels.c \
        Bonobo_Sample_Moody.h

OBJECTS = \
        Bonobo_Sample_Moody-common.o \
        Bonobo_Sample_Moody-skels.o \
        good-mood.o \
        bad-mood.o \
        bonobo-sample-moody-component.o 


bonobo-sample-moody-component: ${CORBA_GENERATED} ${OBJECTS}
        gcc -o $@ ${OBJECTS} `orbit-config --libs server`\ 
                      `gnome-config --libs bonobo gnomeui`

.c.o:   
        gcc -c $< `orbit-config --cflags server` `gnome-config --cflags bonobo` 

$(CORBA_GENERATED):
        orbit-idl --nostubs Bonobo_Sample_Moody.idl \ 
                           `gnome-config --cflags idl`

clean:
        rm -f *~ ${OBJECTS} ${CORBA_GENERATED} \
                 bonobo-sample-moody-component

\end{verbatim}

\section{Writing a client}
We can't do anything with our newly created component if we don't have
some client. However, I won't discuss the writing of a client here;
but - fear not - a test client is included with \gwizard{}, and looks
a bit like this:
\begin{verbatim}
/*
 *  bonobo-sample-moody-client.c
 */
#include <gnome.h>
#include <liboaf/liboaf.h>
#include <bonobo.h>
#include "Bonobo_Sample_Moody.h"

int
main (int argc, char *argv[])
{
        CORBA_ORB orb;
        CORBA_Environment ev;

        BonoboObjectClient *server;

        Bonobo_Unknown moody_object;
        Bonobo_Sample_GoodMood good_mood;
        Bonobo_Sample_BadMood  bad_mood;
        
        gnome_init_with_popt_table ("bonobo-sample-moody-client", 
                                    "0.0.0", argc, argv,
                                    oaf_popt_options, 0, NULL);
        
        if ((orb = oaf_init (argc, argv)) == CORBA_OBJECT_NIL)
                g_error ("could not init orb\n");
        
        if (!bonobo_init (orb, CORBA_OBJECT_NIL, CORBA_OBJECT_NIL))
                g_error ("could not initialize bonobo\n");

        bonobo_activate ();
        
        if (!(server = bonobo_object_activate (
                      "OAFIID:Bonobo_Sample_MoodyComponent", 0)))
                g_error ("failed to create a moody component\n");
        
        CORBA_exception_init (&ev);
        moody_object = BONOBO_OBJREF (server);

        /*
         *  good mood
         */
        good_mood = bonobo_object_client_query_interface 
                (server, "IDL:Bonobo/Sample/GoodMood:1.0", &ev);
        if (BONOBO_EX(&ev))
                g_error ("error querying interface\n");
        else {
                char *msg = NULL;

                g_print ("Q: hey good mood component, how are you?\n");
                msg = Bonobo_Sample_GoodMood_say_hello (good_mood, &ev);
                if (BONOBO_EX(&ev)) 
                        g_warning ("error in say_hello\n");
                else
                        g_print ("A: %s\n", msg);
                
                CORBA_exception_free (&ev);
                CORBA_free (msg);
                
                bonobo_object_release_unref (good_mood, NULL); 
        }

        /* 
         * bad mood 
         */
        bad_mood = bonobo_object_client_query_interface 
                (server, "IDL:Bonobo/Sample/BadMood:1.0", &ev);
        if (BONOBO_EX(&ev))
                g_error ("error querying interface\n");
        else {
                char *msg = NULL;

                g_print ("Q: hey bad mood component, how are you?\n");
                msg = Bonobo_Sample_BadMood_say_hi (bad_mood, &ev);
                if (BONOBO_EX(&ev)) 
                        g_warning ("error in say_hello\n");
                else
                        g_print ("A: %s\n", msg);
                
                CORBA_exception_free (\&ev);
                CORBA_free (msg);
                
                bonobo_object_release_unref (bad_mood, NULL);       
        }
        
                        
        CORBA_exception_free (&ev);
        bonobo_object_unref (BONOBO_OBJECT (server));
        
        return 0;
\end{verbatim}   
The corresponding \verb|Makefile| can be found in the \gwizard{}
distribution. 

I gladly admit the client looks quite complex for the little we do. 
This is partly because small programs have a big ratio of boilerplate
code; another reason is that C-language binding is inherently rather
low-level. You might want to consider the Bonobo Python binding
instead~\cite{pythonbonobo}...

 \section{The impressive results}
Now, let's compile the thing and we're ready for testing:
we have already built our server (right?), and now we have a client too!
\begin{verbatim}
% ./bonobo-sample-moody-client 
Q: hey good mood component, how are you?
A: Hi, how are you!
Q: hey bad mood component, how are you?
A: Grmpff... stop bothering me!
\end{verbatim}
Yes: it worked! You've reached the next level. Sit back and relax. And
enjoy a short Zen moment of Bonobo enlightenment.

\section{Conclusions}
Hopefully, this article has shown that writing Bonobo components is
not hard at all (I hope). Also, I hope it will encourage, {\bf you}, the reader,
to create some really cool Bonobo components -- a little imagination,
a little transpiration and a little elisp can do wonders...

\gwizard{} is quite handy; however I need to stress again that a
code generator can never be a substitute for understanding. Study the
generated code. Understand it.

\section{Future work on gwizard}
\gwizard{} is currenly alpha level software and probably still
contains a number of bugs. It seems very useful though.

Anyway, apart from fixing bugs, there are some features I may or may not add.
\begin{itemize}
  \item It would be easy to extend \gwizard{} to create the boilerplate for the
  IDL-interfaces as well, and maybe the
  \verb|Makefile|/\verb|Makefile.am|;
  \item It would also be quite easy to generate the code for
  shared-library components as well;
  \item Some extra magic could be used to further ease the writing of
  specific Bonobo components (such as Monikers);
  \item There should be some shell scripts to run the elisp in batch
  mode, which is more convenient for people that don't use Emacs;
  \item Also, the current \gwizard{} support Bonobo version 1.x; work
  is already underway to support the GNOME2/Bonobo2 platform as well.
\end{itemize} 
\bibliographystyle{plain} \flushleft
\bibliography{books}
\end{document}
%
% end of bonobo-easy.tex
%
